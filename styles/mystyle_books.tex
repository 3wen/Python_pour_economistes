\usepackage{etex} %vers l'infini et au-dela
%\reserveinserts{28} % (et plus loin encore)

\usepackage[utf8]{inputenc} 
\usepackage[T1]{fontenc} % pour taper les lettres accentuees
\usepackage[french, english]{babel} %le rapport est en franÁais 
\usepackage{amsthm,amssymb,amsbsy,amsmath,amsfonts,amssymb,amscd,mathrsfs}
%Mise en page
\usepackage{geometry} %pour la modification des marges
\usepackage{fancyhdr} %pour modification des pieds de page
\usepackage{lastpage} %numero de la derniËre page
\usepackage{lscape} % Pour pouvoir activer le mode landscape
\usepackage{lmodern}
%Images, figures, etc.
\usepackage{booktabs} %tableaux
\usepackage{float}	%pour forcer le placement des images.
\usepackage{graphicx} %pour afficher des images
\usepackage{longtable} %tables sur plusieurs pages
\usepackage{animate} %transformer des gifs
\usepackage{caption}
%\usepackage{subcaption}
\captionsetup{labelformat=simple}
\usepackage[scriptsize]{subfigure}
\usepackage{multirow} 			%fusionner lignes
\usepackage{tikz}
\usepackage{adjustbox}
\usetikzlibrary{arrows,positioning}
\usetikzlibrary{mindmap,trees,shadows,backgrounds}
\usepackage{tabularx}
%Code
\usepackage{verbatim}%insertion de code
\usepackage{listings}
%Polices, format,couleurs
\usepackage{dsfont} % Pour les lettres mathematiques

\usepackage[nottoc]{tocbibind}

\usepackage{hyperref} %pour que les references soient des liens hypertextes
\usepackage{natbib}
\usepackage{bibentry}
%\usepackage{color}
\usepackage{xcolor,colortbl}
\definecolor{light-gray}{gray}{0.85}
\usepackage{ragged2e} % Pour justifier
%Symboles, theoremes
\newcommand{\iid}{\stackrel{\mathrm{iid}}{\sim}}
\newtheorem{theorem}{Theorem}[part]
\newtheorem{lemma}[theorem]{Lemma}
\newtheorem{hypothese}{Hypoth\`ese}
\newtheorem{corollary}[theorem]{Corollary}
\usepackage{enumerate}%listes
\usepackage[subnum]{cases}% cas numerotes
%\newtheorem{laRemarque}{Remarque}[section]
 \newtheorem{rmarque}{Remarque}[section]
\newtheorem{exmp}{Exemple}[section]
\numberwithin{equation}{section}

\usepackage[official,right]{eurosym} % symbols euro
\usepackage{gensymb} % symbole degre \degre
\usepackage{rotating}
\usepackage{multicol}
% Bibliographie
%\usepackage{bib entry}
%\nobibliography*
%\let\newblock\relax

%\usepackage[notocbib]{apacite}
%\usepackage[style=apa]{biblatex}
%\usepackage[round]{natbib}
%\usepackage[style=authoryear]{biblatex}
%\usepackage[backend=bibtex, style=authoryear-comp]{biblatex}
%\usepackage{biblatex}
\renewcommand*{\bibfont}{\scriptsize} % Pour avoir la biblio en plus petit
%\usepackage{apacite}
%\bibliographystyle{apacite}

%\titlegraphic{
%\centering
%\includegraphics[height=2em]{/Users/ewengallic/Dropbox/Universite_Aix_Marseille/Enseignements/template/assets/images/logo_amu_rvb_blanc.png}\hskip 3em
%%\colorbox{white}{\includegraphics[height=3em]{../../images/logo_feg.png}}
%\includegraphics[height=2em]{/Users/ewengallic/Dropbox/Universite_Aix_Marseille/Enseignements/template/assets/images/logo_feg_blanc.png}\hskip 3em
%%\colorbox{white}{\includegraphics[height=3em]{../../images/logo_amse.png}}
%\includegraphics[height=2em]{/Users/ewengallic/Dropbox/Universite_Aix_Marseille/Enseignements/template/assets/images/logo_amse.png}
%}

\justifying %on justifie le texte du document


% Pour les codes Python
\lstset{
language=Python,
basicstyle=\small\ttfamily,
commentstyle=\ttfamily\color{gray},
backgroundcolor=\color{white},
showspaces=false,
showstringspaces=false,
showtabs=false,
tabsize=4,
captionpos=b,
breaklines=true,
breakatwhitespace=false,
title=\lstname,
escapeinside={},
keywordstyle={},
morekeywords={},
literate=%
         {á}{{\'a}}1
         {à}{{\`a}}1
         {^a}{{\^a}}1
         {í}{{\'i}}1
         {ï}{{\"\i}}1
         {é}{{\'e}}1
         {è}{{\`e}}1
         {ê}{{\^e}}1
         {ë}{{\"e}}1
         {ý}{{\'y}}1
         {ú}{{\'u}}1
         {ù}{{\`u}}1
         {ó}{{\'o}}1
         {ô}{{\^o}}1         
         {ě}{{\v{e}}}1
         {š}{{\v{s}}}1
         {č}{{\v{c}}}1
         {ř}{{\v{r}}}1
         {ž}{{\v{z}}}1
         {ď}{{\v{d}}}1
         {ť}{{\v{t}}}1
         {ň}{{\v{n}}}1                
         {ů}{{\r{u}}}1
         {Á}{{\'A}}1
         {Í}{{\'I}}1
         {É}{{\'E}}1
         {Ý}{{\'Y}}1
         {Ú}{{\'U}}1
         {Ó}{{\'O}}1
         {Ě}{{\v{E}}}1
         {É}{{\'E}}1
         {È}{{\`E}}1
         {Ê}{{\^E}}1
         {Š}{{\v{S}}}1
         {Č}{{\v{C}}}1
         {Ř}{{\v{R}}}1
         {Ž}{{\v{Z}}}1
         {Ď}{{\v{D}}}1
         {Ť}{{\v{T}}}1
         {Ň}{{\v{N}}}1                
         {Ů}{{\r{U}}}1 
}




%\graphicspath{{../images}}


\usepackage{array,dcolumn}
\newcolumntype{C}[1]{>{\centering\arraybackslash}m{#1}}
\newcolumntype{R}[1]{>{\raggedleft\arraybackslash}m{#1}}
\newcolumntype{L}[1]{>{\raggedright\arraybackslash}m{#1}}




\definecolor{vert}{RGB}{0,157,87}
\definecolor{Lime}{RGB}{191,255,0}
\definecolor{limegreen}{RGB}{50,205,50}
\definecolor{vertsombre}{RGB}{27,68,21}
\definecolor{vertclair}{HTML}{06D6A0}

%
\definecolor{Camel}{RGB}{193,154,107}
\definecolor{rouge}{HTML}{EF476F}
\definecolor{orange}{RGB}{199,118,34}
%
\definecolor{ProcessYellow}{RGB}{255, 239, 0}
\definecolor{jauneAMSE}{RGB}{240, 171, 0}
\definecolor{jaune}{HTML}{FFD166}


%
\definecolor{bleu}{HTML}{118AB2}
\definecolor{LBlue}{RGB}{173, 216, 230}
\definecolor{bleusombre}{RGB}{18, 37, 67}
\definecolor{bleufonce}{RGB}{91,112,170}
\definecolor{bleuAMSE}{RGB}{0,101,189}
%
\definecolor{DPurple}{RGB}{48,25,52}
\definecolor{Indigo}{RGB}{75, 0, 130}
\definecolor{Periwinkle}{RGB}{204, 204, 255}
\definecolor{ElectricViolet}{RGB}{143, 0, 255}
\definecolor{AfricanViolet}{RGB}{178, 132, 190}
\definecolor{ChineseViolet}{RGB}{133, 96, 136}
\definecolor{Grape}{RGB}{111, 45, 168}
\definecolor{RussianViolet}{RGB}{50, 23, 77}
\definecolor{EnglishViolet}{RGB}{86, 60, 92}
%

\definecolor{bottomcolour}{rgb}{0.32,0.3,0.38}
\definecolor{middlecolour}{rgb}{0.08,0.08,0.16}



%\hypersetup{
%    colorlinks = true,
%    linkcolor = ProcessYellow
%}




\newcommand*\grasO[1]{\textbf{\textcolor{Camel}{#1}}} 
\newcommand*\grasR[1]{\textbf{\textcolor{rouge}{#1}}}
\newcommand*\grasV[1]{\textbf{\textcolor{vertclair}{#1}}} 
\newcommand*\grasB[1]{\textbf{\textcolor{bleu}{#1}}} 
\newcommand*\grasJ[1]{\textbf{\textcolor{jaune}{#1}}} 
\newcommand*\grasP[1]{\textbf{\textcolor{AfricanViolet}{#1}}} 

\definecolor{rcp26}{RGB}{39, 55, 122}
\definecolor{rcp45}{RGB}{112, 159, 200}
\definecolor{rcp60}{RGB}{222, 99, 43}
\definecolor{rcp85}{RGB}{205, 16, 32}

\newcommand*\grasRCPa[1]{\textbf{\textcolor{rcp26}{#1}}}
\newcommand*\grasRCPb[1]{\textbf{\textcolor{rcp45}{#1}}} 
\newcommand*\grasRCPc[1]{\textbf{\textcolor{rcp60}{#1}}} 
\newcommand*\grasRCPd[1]{\textbf{\textcolor{rcp85}{#1}}}

\usepackage{tcolorbox}

\makeatletter
\let\@@magyar@captionfix\relax
\makeatother


\definecolor{deepblue}{RGB}{0,110,160}



\usepackage{xparse}
\NewDocumentCommand\afun{v}{%
\texttt{#1}\index[fonctions]{#1@\ifun{#1}}%
}
\NewDocumentCommand\afunDeux{v}{%
\index[fonctions]{#1@\ifun{#1}}%
}


\NewDocumentCommand{\ifun}{v}{\texttt{#1}}


 \newcounter{countremarque}


\newenvironment{remarque}{%
 \refstepcounter{countremarque}
    \begin{tcolorbox}[width=\linewidth, colback=blue!3, boxrule=0.5pt,arc=0pt,title = Remarque \thecountremarque]
    }%
    {
    \end{tcolorbox}
    }
\numberwithin{countremarque}{section}


\newcounter{exercices}[section]

\definecolor{shadecolorex}{HTML}{CFE3D1}
\makeatletter
\newenvironment{exframe}{%
 \def\at@end@of@exframe{}%
 \ifinner\ifhmode%
  \def\at@end@of@exframe{\end{minipage}}%
  \begin{minipage}{\columnwidth}%
 \fi\fi%
 \def\FrameCommand##1{\hskip\@totalleftmargin \hskip-\fboxsep
 \colorbox{shadecolorex}{##1}\hskip-\fboxsep
     % There is no \\@totalrightmargin, so:
     \hskip-\linewidth \hskip-\@totalleftmargin \hskip\columnwidth}%
 \MakeFramed {\advance\hsize-\width
   \@totalleftmargin\z@ \linewidth\hsize
   \@setminipage}}%
 {\par\unskip\endMakeFramed%
 \at@end@of@exframe}
\makeatother




\hypersetup{
bookmarks=true,
pdfauthor={Ewen Gallic},
unicode=false,
pdftoolbar=true,
pdfmenubar=true,
pdffitwindow=false
pdfnewwindow=true,
colorlinks=true,
linkcolor=bleuAMSE,
citecolor=bleuAMSE,
filecolor=bleuAMSE,
urlcolor=bleuAMSE,}
